\hypertarget{system-model}{%
\chapter{System Model}\label{chapter:model}}

This section details the model currently used to describe Bitswap in
this work. \texttt{\textbackslash{}cref\{network-graph\}} models the
IPFS network as a graph; \texttt{\textbackslash{}cref\{reputation\}}
mathematically describes the peer-wise reputations and user
interactions; and \texttt{\textbackslash{}cref\{game-formulation\}}
formulates the problem as a game.

\hypertarget{network-graph}{%
\subsection{Network Graph}\label{network-graph}}

We model an IPFS swarm as a network \texttt{\textbackslash{}Network}~of
$\abs{\Network}$ users. The graphical representation consists of

\begin{itemize}
\item
  \emph{nodes} representing users, and
\item
  \emph{unweighted, undirected edges}, each of which represents a
  peering between two users.
\end{itemize}

A user's \emph{neighborhood} is their set of peers, i.e.~the set of
nodes that the user is connected to by an edge. User $i$'s
neighborhood is denoted by $\Nbhd{i}$, where
$\Nbhd{i} \subseteq \Network$.

\hypertarget{reputation}{%
\subsection{Reputation}\label{reputation}}

We break Bitswap interactions into discrete rounds, with a single round
denoted by a nonnegative integer $t$. The following two points
describe the way data distribution takes place in this Bitswap model.
Each of these points simplifies the problem from the real-world
scenario.

\begin{enumerate}
\def\labelenumi{\arabic{enumi}.}
\item
  Each user $j$ distributes exactly $B_j$ bits, where $B_j > 0$,
  to each of their peers in a given round (and has sufficient resources
  to do so).
\item
  All users always have unique data that all of their peers want. So,
  when a user allocates $b$ bits to a particular peer, that user has
  at least $b$ bits that the peer wants.
\end{enumerate}

We define $b_{ij}^t$ as the total number of bits sent from user $i$
to peer $j$ from round $0$ to $t-1$. User $i$ maintains a
Bitswap \emph{ledger} $l_{ij}^t$, for each of its peers
$j \in \Nbhd{i}$, that stores the amount of data exchanged in both
directions, i.e. $l_{ij}^t = (b_{ij}^t, b_{ji}^t)$.

Now we define the \emph{debt ratio} $d_{ji}$ from user $i$ to peer
$j$ as

\[d_{ji}^t = \frac{b_{ij}^{t-1}}{b_{ji}^{t-1}\:+\:1}\]

$d_{ji}^t$ can be thought of as peer $i$'s reputation from the
perspective of user $j$. This reputation is then considered by user
$j$'s \emph{reputation function}
$S_j(d_{ji}^t, \mathbf{d}_j^{-i,t}) \in \{0, 1\}$, where
$\mathbf{d}_j^{-i,t} = (d_{jk}^t \mid \forall k \in \Nbhd{j}, k \neq i)$
is the vector of debt ratios for all of user $j$'s peers in round
$t$ \emph{excluding} peer $i$. The reputation function considers the
relative reputation of peer $i$ to the rest of $j$'s peers, and
returns a weight for peer $i$. This weight is used to determine what
proportion of $j$'s resources to allocate to peer $i$ in round
$t$. This means that the following must hold:

\[\sum_{i \in \Nbhd{j}} S_j(d_{ji}^t, \mathbf{d}_j^{-i,t}) = 1\]

In other words, the sum of the weights that $j$ assigns to its peers
must be $1$ (otherwise, $j$ would allocate more than its available
bandwidth to its peers).

Given all of this, we can calculate $b_{ji}^{t+1}$ given $b_{ji}^t$,
$d_{ji}^t$, and `mathbf\{d\}\_j\^{}\{-i,t\}`:

\[b_{ji}^{t+1} = b_{ji}^{t} + S_j(d_{ji}^t, \mathbf{d}_j^{-i,t}) \times B_j\]

\hypertarget{game-formulation}{%
\subsection{Game Formulation}\label{game-formulation}}

The game model presented here is the product of multiple iterations that
approached a balance between the accuracy of the model to actual Bitswap
interactions model complexity. Previous modeling approaches included
tools from evolutionary game theory and statistical mechanics on the
high complexity end, and repeated games on the low accuracy end. While
the current model uses a repeated game model as well, the strategy space
has been modified to better fit the Bitswap
scenario.{[}\^{}old-model-link{]}

All users in the network participate in the Bitswap game. The players in
this game are the users in the IPFS network, and each player's strategy
is the reciprocation function they choose to assign weights to their
peers. The strategy space, then, is the space containing
functions of the form
$S_j(d_{ji}^t, \mathbf{d}_j^{-i,t}) \in \{0, 1\}$. The Bitswap game is
an \emph{infinitely repeated}, \emph{incomplete information} game in
which users exchange data. One game takes place between each pair of
peers for each round $t$. The \emph{players} are the IPFS users in the
network \texttt{\textbackslash{}Network}. The utility of player $i$ in
round $t$ is the sum of all of the data that $i$ is sent by its
peers in that round:

\[u_i^t = \sum_{j \in \Nbhd{i}} (b_{ji}^t - b_{ji}^{t-1})\]

The total utility of a peer over the entire repeated game, then, can be
expressed in the following ways:

\begin{align*}
U_i &= \sum_{t=0}^{\infty} u_i^t \\
U_i &= \sum_{j \in \Nbhd{i}} b_{ji}^\infty
\end{align*}

The latter representation can be most directly thought of as the total
amount of data that peer $i$ receives from all of its peers over the
entire duration of the Bitswap game.

\hypertarget{discussion}{%
\subsection{Discussion}\label{discussion}}

\texttt{TODO}

(Complexity of the model (roughly/layman's terms); brief glimpse of it
compared to an evolutionary game model and maybe my old repeated game
model; how this compromises between those two to be a realistic but
manageably complex model.)

The points in this section should then be related to each of my
contributions: analytical, simulation, and implementation work. Would
make sense to do that at the beginning of the next section, probably, or
whatever section introduces those things.
