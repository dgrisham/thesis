\hypertarget{introduction}{%
\chapter{Introduction}\label{chapter:introduction}}

The Internet is perhaps the largest and most long-running network that has
ever existed. Unfortunately, it predominantly runs on the outdated
hypermedia distribution protocol HTTP. The goal of the InterPlanetary
File System (IPFS) is to upgrade the Internet to a distributed
peer-to-peer system, thereby making it more robust and permanent. This
new Internet would be a network of peers, as opposed to clients and
servers, all sharing data between one another. In order for such a
system to thrive, users must be cooperative and willing to share data
with their peers. The goal of this project is to analyze the resource
allocation options of peers interacting in an IPFS network. A
combination of analytical and empirical methods will be used to glean
insights into the generally intractable allocation decisions that users
are presented with when participating in an IPFS network.

\hypertarget{ipfs}{%
\subsection{IPFS}\label{ipfs}}

IPFS is a peer-to-peer hypermedia distribution protocol developed by
Protocol Labs. It is a content-addressed, versioned filesystem. While a
variety of use cases exist for such a protocol, the most ambitious goal
of the project is to replace HTTP as the primary file exchange protocol
used in the Internet. This could ultimately result in the
decentralization of the Internet.

IPFS synthesizes various technologies developed since the Internet's
inception. These technologies include Git, BitTorrent, distributed hash
tables (e.g.~Kademlia), and self-certified filesystems. The IPFS stack
is shown in Figure \href{}{img:ipfs-stack}. One way to conceptualize an
IPFS network is as a Git repository shared within a torrent-esque swarm.

\leavevmode\vadjust pre{\hypertarget{img:ipfs-stack}{}}%
figures/ipfs\_stack.png

\textbf{The IPFS Stack -\/- Bitswap is the exchange layer}

\hypertarget{bitswap}{%
\subsection{Bitswap}\label{bitswap}}

Bitswap is the block exchange protocol of IPFS. The most direct
inspiration of this submodule is the BitTorrent peer-to-peer file
distribution protocol. Bitswap is the layer of IPFS that incentivizes
users to share data. A Bitswap \emph{reciprocation function} determines
which peers to send data to, and in what relative quantities. The input
to the reputation function is a set of metrics that may be used to weigh
peers -- e.g.~peer bandwidth, reputation, and/or location. The output is
a set of weights, one for each peer, that assign the relative resource
allocations for the peers. These weights are periodically recalculated
to reflect changes in both the network and peer behavior.

\hypertarget{objectives}{%
\subsection{Objectives}\label{objectives}}

\texttt{This\ section\ will\ likely\ need\ to\ be\ updated/replaced,\ as\ it\ was
copied\ directly\ from\ the\ proposal.\ At\ the\ very\ least,\ I\ think\ it\ might
need\ to\ be\ changed\ to\ be\ past-tense\ instead\ of\ future}

For this project, I will take the initial steps toward understanding the
behavior of users in an IPFS network as predicted by game theoretical
models. This will involve a combination of analytical and empirical
analyses, along with implementation of these ideas in the IPFS software.
The analytical work will focus on repeated games and, potentially,
evolutionary games, while the empirical work will take a
simulation-based approach. I intend to use these methods to classify
various Bitswap reciprocation functions and determine useful allocation
behavior under certain conditions.

\hypertarget{related-work}{%
\subsection{Related Work}\label{related-work}}

\hypertarget{bittorrent}{%
\subsubsection{BitTorrent}\label{bittorrent}}

Much work has been done to understand the incentives and gameability of
the BitTorrent peer-to-peer file sharing protocol. Two research-based
BitTorrent clients that implement alternative bandwidth sharing strategies,
BitTyrant and BitThief, demonstrated strategies that disproved the standard
BitTorrent strategy as a Nash Equilibrium. BitTyrant adjusts the bandwidth
shared with a peer in order to receive the same share of the peer's bandwidth
while uploading as little as possible with that peer. BitThief leverages
Bitswap's optimistic unchoking algorithm to successfully freeride off of the
network.

PropShare extends these findings by introducing an alternative to the standard
BitTorrent strategy that corrects for weaknesses presented by BitTyrant and
BitThief. In BitTorrent, a user ranks their peers in terms of the amount of
bandwidth those peers have provided to the user, takes the top $N$, and splits
its bandwidth equally among the $N$ best peers. PropShare ranks the peers similarly,
but splits bandwidth *proportionally* among the peers based on the relative
bandwidth allocations given to the user.

IPFS/Bitswap have marked differences from BitTorrent. For instance, a given
torrent is a static set of file shared among a network of peers that is specific
to that torrent (called a 'swarm'). In IPFS, on the other hand, data may be added
to the swarm at any given time, and the average lifespan of a user in a swarm
will often be much longer than that in a torrent swarm.
